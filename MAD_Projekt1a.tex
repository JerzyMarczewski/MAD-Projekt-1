\documentclass{article}

% żeby użyć polskiego
\usepackage[T1]{fontenc}
\usepackage[polish]{babel}
\usepackage[utf8]{inputenc}

% żeby użyć matmy
\usepackage{amsmath}

% żeby pokazać kod
\usepackage{listings}

% żeby wklejać zdjęcia
\usepackage{graphicx}

\title{Analiza skupień nakładów inwestycyjnych w Polsce w latach 2010-2020}
\author{Jerzy Marczewski, Dawid Rozumkiewicz, Wojciech Pietraszuk, Karol Hetmański, Przemysław Chachaj}
\date{}
\begin{document}

\maketitle

\section{Streszczenie}

\section{Słowa kluczowe}

\section{Wprowadzenie}

\section{Przedmiot badania}
    \begin{itemize}
        \item Cel i zakres badania
        \item Przegląd literatury (min. 1 cytowanie powiązane tematycznie i krótki opis co było badane)
        \item Zmienne wybrane do analizy (opis i uzasadnienie zmiennych oraz podział na stymulanty/destymulanty) minimum sześć zmiennych
        \item Wstępna analiza danych
        \begin{itemize}
            \item statystyki opisowe (przynajmniej: średnia, mediana, minimum, maksimum, odchylenie standardowe, skośność)
            \item podstawowa wizualizacja np. boxplot, histogramy
            \item braki danych, czy występują i jak je obsłużono
            \item obserwacje odstające i w jaki sposób je obsłużono
        \end{itemize}
    \end{itemize}

\section{Opis metod}
    \begin{itemize}
        \item wzory wraz z opisami oznaczeń
        \item cytowanie pracy w której zaproponowano metodę/ewentualnie pracy, w której użyto metodę
    \end{itemize}

\section{Rezultaty (w postaci tabelarycznej i/lub graficznej oraz omówienie wyników)}
\section{Podsumowanie (ocena realizacji celu, odniesienie do pozycji z przeglądu literatury)}
\section{Bibliografia}

\end{document}
