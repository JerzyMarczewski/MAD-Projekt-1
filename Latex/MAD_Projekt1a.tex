\documentclass{article}

% żeby użyć polskiego
\usepackage[T1]{fontenc}
\usepackage[polish]{babel}
\usepackage[utf8]{inputenc}

% żeby użyć matmy
\usepackage{amsmath}

% żeby pokazać kod
\usepackage{listings}

% żeby wklejać zdjęcia
\usepackage{graphicx}

% tabelki
\usepackage{array}

\title{Analiza skupień zestawu parametrów samochodów amerykańskich, europejskich i japońskich wyprodukowanych w latach 1970-1982}
% \author{Jerzy Marczewski, Dawid Rozumkiewicz, Wojciech Pietraszuk, Karol Hetmański, Przemysław Chachaj}
\author{\fontsize{11}{13}\selectfont Jerzy Marczewski, \newline Dawid Rozumkiewicz, Wojciech Pietraszuk, Karol Hetmański, Przemysław Chachaj}
\author{
  Jerzy Marczewski\\
  \and
  Dawid Rozumkiewicz\\
  \and
  Wojciech Pietraszuk\\
  \and
  Karol Hetmański\\
  \and
  Przemysław Chachaj\\
}
\date{}
\begin{document}

\maketitle

\section{Streszczenie}

\section{Słowa kluczowe}
    \begin{itemize}
        \item klaster 
        \item klateryzacja - metoda klasyfikacji bez nadzoru (ang. usupervised learning), która grupuje elementy na względnie jednorodne klasy
        \item mpg - \textit{(ang. miles per gallon)} mile na galon
        \item objętość skokowa cylindra - różnica pomiędzy maksymalną a minimalną objętością cylindra
        \item objętość skokowa silnika - iloczyn objętości skokowej cylindra i liczby cylindrów
        \item hp - \textit{(ang. Horsepower)} 
        \item metoda k-średnich
        \item metoda Warda
        \item SOM - \textit{(ang. Self-organizing map)} sieć Kohonena 
        \item BMU - \textit{(ang. best matching unit)} najlepiej dopasowana jednostka
    \end{itemize}


\section{Wprowadzenie}

\section{Przedmiot badania}
    \subsection{Cel i zakres badania} 
    \subsection{(min. 1 cytowanie powiązane tematycznie i krótki opis co było badane)}
    \subsection{Zmienne wybrane do analizy (opis i uzasadnienie zmiennych oraz podział na stymulanty/destymulanty) minimum sześć zmiennych}
    Do analizy wybrano następujące cechy diagnostyczne:
    \begin{itemize}
        \item ${X_1}$ - mpg
        \item ${X_2}$ - liczba cylidnrów 
        \item ${X_3}$ - objętość skokowa silnika
        \item ${X_4}$ - hp
        \item ${X_5}$ - waga (podana w funtach)
        \item ${X_6}$ - przyspieszenie
        \item ${X_7}$ - model (rok modelowy auta wyrażony w dwóch ostanich cyfrach roku)
        \item ${X_8}$ - kraj pochodzenia (1 - Stany Zjednoczone, 2 - Europa, 3 - Japonia)
    \end{itemize}
    
    \subsection{Wstępna analiza danych}
        \subsubsection*{Statystyki opisowe}
        Wyniki średniej, mediany, minimum, maksimum, odchylenia standardowego, skośności dla każdej z cech zaokrąglone do dwóch 
        miejsc po przeciunku:
        \begin{center}
            \begin{tabular}{ |c|c|c|c|c|c|c|c|c|c| } 
                \hline
                & ${X_1}$ & ${X_2}$ & ${X_3}$ & ${X_4}$ & ${X_5}$ & ${X_6}$ & ${X_7}$ & ${X_8}$ \\
                \hline
                średnia & 20.76  & 5.81 & 216.09 & 112.91 & 3139.32 & 15.31 & 74.35 & 1.46 \\
                \hline
                mediana & 19.2 & 6.0 & 225.0 & 100.0 & 3139.0 & 15.5 & 74.0 & 1.0 \\
                \hline
                min & 9 & 4 & 72 & 46 & 1613 & 9 & 70 & 1  \\
                \hline
                max & 43.1 & 8.0 & 455.0 & 230.0 & 5140.0 & 22.1 & 79.0 &  3.0  \\
                \hline
                odchylenie standardowe & 6.62 & 1.78 & 109.93 & 41.64 & 900.74 & 2.80 & 2.93 &  0.74  \\
                \hline
                skośność & 0.71 & 0.18 & 0.34 & 0.79 & 0.24 & 0.06 & 0.04 &  1.21 \\
            \hline
            \end{tabular}
        \end{center}
        \subsubsection*{podstawowa wizualizacja np. boxplot, histogramy}
        \subsubsection*{braki danych, czy występują i jak je obsłużono}
        Braki danych występują jedynie dla cechy hp. Wszystkie samochody z brakiem danych zostały usunięte z danych za pomocą skryptu w R.
        \subsubsection*{obserwacje odstające i w jaki sposób je obsłużono}

\section{Opis metod}
    \subsection{wzory wraz z opisami oznaczeń}
        \subsubsection*{Metoda k-średnich}
        Celem metody jest przypisanie do wektorów $r_i$ n wymiarowych wektorów danych,
        przy jak najmniejszym średnim błędzie kwantyzacji.
        Średni błąd kwantyzacji opisany jest wzorem:
            \begin{equation*}
                D = \frac{1}{K}\sum_{i=1}^{K} d(x_i, r)
            \end{equation*}
            \begin{itemize}
                \item K - liczba elementów ${x_i}$ przypisanych do wektora r
                \item d - miara błędu kwantyzacji, najczęściej błąd kwadratowy opisany wzorem:
                \begin{equation*}
                    d(x,r) = \sum_{j=1}^{n} (x_j - r_j)^2
                \end{equation*}
            \end{itemize}
        \subsubsection*{Metoda Warda}
            Odległość nowego skupienia od każdego pozostałego: 
            \begin{equation*}
                D_{pr} = a_{1}\cdot d_{pr} + a_{2}\cdot d_{qr} + b\cdot d_{pq}
            \end{equation*}
            \begin{itemize}
                \item r - numery skupień różne od p i q
                \item $D_{pr}$ - odległość nowego od skupienia r
                \item $d_{pr}$ - odległość pierwotnego skupienia p od skupienia r
                \item $d_{qr}$ - odległość pierwotnego skupienia q od skupienia r
                \item $d_{pq}$ - wzajemna odległość pierwotnych skupień p i q
                \item $a_{1} = \frac{n_{p} + n_{r}}{n_{p} + n_{q} + n_{r}}$, $a_{2} = \frac{n_{q} + n_{r}}{n_{p} + n_{q} + n_{r}}$, $b = \frac{-n_{r}}{n_{p} + n_{q} + n_{r}}$
                \item n - liczebność pojedyńczych obiektów w poszczególnych obiektach
            \end{itemize}
        \subsubsection*{SOM}
        Wzór aktualizowania neuronu v z wagą wektora $W_v(s)$:
            \begin{equation*}
                W_v(s+1) = W_v(s) + \theta(u,v,s) \cdot \alpha(s) \cdot (D(t)-W_v(s))
            \end{equation*}
            \begin{itemize}
                \item s - obecna iteracja
                \item t - indeks docelowego wektora danych wejściowych w zbiorze danych wejściowych D 
                \item D(t) - docelowy wektor danych wejściowych
                \item v - indeks wektora w mapie
                \item $W_v$ - aktualny wektor wagi węzła v
                \item u - to indeks BMU na mapie
                \item ${\theta(u,v,s)}$ - jest ograniczeniem ze względu na odległość od BMU, zwykle nazywaną funkcją sąsiedztwa
            \end{itemize}
    \subsection{cytowanie pracy w której zaproponowano metodę/ewentualnie pracy, w której użyto metodę}

\section{Rezultaty (w postaci tabelarycznej i/lub graficznej oraz omówienie wyników)}
\section{Podsumowanie (ocena realizacji celu, odniesienie do pozycji z przeglądu literatury)}
\section{Bibliografia}
    \begin{itemize}
        \item Algorithm AS 136: A K-Means Clustering Algorithm - J. A. Hartigan and M. A. Wong
        \item www.statystyka.az.pl
        \item www.wikipedia.org
    \end{itemize}

\end{document}
