\documentclass{article}

% żeby użyć polskiego
\usepackage[T1]{fontenc}
\usepackage[polish]{babel}
\usepackage[utf8]{inputenc}

% żeby użyć matmy
\usepackage{amsmath}

% żeby pokazać kod
\usepackage{listings}

% żeby wklejać zdjęcia
\usepackage{graphicx}

% tabelki
\usepackage{array}

\title{Analiza skupień zestawu samochodów amerykańskich, europejskich i japońskich wyprodukowanych w latach 1970-1982}
% \author{Jerzy Marczewski, Dawid Rozumkiewicz, Wojciech Pietraszuk, Karol Hetmański, Przemysław Chachaj}
\author{\fontsize{11}{13}\selectfont Jerzy Marczewski, \newline Dawid Rozumkiewicz, Wojciech Pietraszuk, Karol Hetmański, Przemysław Chachaj}
\author{
  Jerzy Marczewski\\
  \and
  Dawid Rozumkiewicz\\
  \and
  Wojciech Pietraszuk\\
  \and
  Karol Hetmański\\
  \and
  Przemysław Chachaj\\
}
\date{}
\begin{document}

\maketitle

\section{Streszczenie}

\section{Słowa kluczowe}

\section{Wprowadzenie}

\section{Przedmiot badania}
    \subsection{Cel i zakres badania} 
    \subsection{(min. 1 cytowanie powiązane tematycznie i krótki opis co było badane)}
    \subsection{Zmienne wybrane do analizy (opis i uzasadnienie zmiennych oraz podział na stymulanty/destymulanty) minimum sześć zmiennych}
    Do analizy wybrano następujące cechy diagnostyczne:
    \begin{itemize}
        \item ${X_1}$ - mpg
        \item ${X_2}$ - liczba cylidnrów 
        \item ${X_3}$ - objętość skokowa silnika
        \item ${X_4}$ - hp
        \item ${X_5}$ - waga (podana w funtach)
        \item ${X_6}$ - przyspieszenie
        \item ${X_7}$ - model (rok modelowy auta wyrażony w dwóch ostanich cyfrach roku)
        \item ${X_8}$ - kraj pochodzenia (1 - Stany Zjednoczone, 2 - Europa, 3 - Japonia)
    \end{itemize}
    
    \subsection{Wstępna analiza danych}
        \subsubsection*{Statystyki opisowe}
        Wyniki średniej, mediany, minimum, maksimum, odchylenia standardowego, skośności dla każdej z cech zaokrąglone do dwóch 
        miejsc po przeciunku:
        \begin{center}
            \begin{tabular}{ |c|c|c|c|c|c|c|c|c|c| } 
                \hline
                & ${X_1}$ & ${X_2}$ & ${X_3}$ & ${X_4}$ & ${X_5}$ & ${X_6}$ & ${X_7}$ & ${X_8}$ \\
                \hline
                średnia & 20.76  & 5.81 & 216.09 & 112.91 & 3139.32 & 15.31 & 74.35 & 1.46 \\
                \hline
                mediana & 19.2 & 6.0 & 225.0 & 100.0 & 3139.0 & 15.5 & 74.0 & 1.0 \\
                \hline
                min & 9 & 4 & 72 & 46 & 1613 & 9 & 70 & 1  \\
                \hline
                max & 43.1 & 8.0 & 455.0 & 230.0 & 5140.0 & 22.1 & 79.0 &  3.0  \\
                \hline
                odchylenie standardowe & 6.62 & 1.78 & 109.93 & 41.64 & 900.74 & 2.80 & 2.93 &  0.74  \\
                \hline
                skośność & 0.71 & 0.18 & 0.34 & 0.79 & 0.24 & 0.06 & 0.04 &  1.21 \\
            \hline
            \end{tabular}
        \end{center}
        \subsubsection*{podstawowa wizualizacja np. boxplot, histogramy}
        \subsubsection*{braki danych, czy występują i jak je obsłużono}
        Braki danych występują jedynie dla cechy hp. Wszystkie samochody z brakiem danych zostały usunięte z danych za pomocą skryptu w R.
        \subsubsection*{obserwacje odstające i w jaki sposób je obsłużono}

\section{Opis metod}
    \subsection{wzory wraz z opisami oznaczeń}
        \subsubsection*{Metoda k-średnich}
        \subsubsection*{Metoda Warda}
        \subsubsection*{SOM}
    \subsection{cytowanie pracy w której zaproponowano metodę/ewentualnie pracy, w której użyto metodę}

\section{Rezultaty (w postaci tabelarycznej i/lub graficznej oraz omówienie wyników)}
\section{Podsumowanie (ocena realizacji celu, odniesienie do pozycji z przeglądu literatury)}
\section{Bibliografia}
    \begin{itemize}
        \item Algorithm AS 136: A K-Means Clustering Algorithm - J. A. Hartigan and M. A. Wong
        \item www.statystyka.az.pl
        \item www.wikipedia.org
    \end{itemize}

\end{document}
