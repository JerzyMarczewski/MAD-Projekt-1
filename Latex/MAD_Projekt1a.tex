\documentclass{article}

% żeby użyć polskiego
\usepackage[T1]{fontenc}
\usepackage[polish]{babel}
\usepackage[utf8]{inputenc}

% żeby użyć matmy
\usepackage{amsmath}

% żeby pokazać kod
\usepackage{listings}

% żeby wklejać zdjęcia
\usepackage{graphicx}

% tabelki
\usepackage{array}

\title{Analiza skupień nakładów inwestycyjnych w Polsce w latach 2010-2020}
\author{Jerzy Marczewski, Dawid Rozumkiewicz, Wojciech Pietraszuk, Karol Hetmański, Przemysław Chachaj}
\date{}
\begin{document}

\maketitle

\section{Streszczenie}

\section{Słowa kluczowe}

\section{Wprowadzenie}

\section{Przedmiot badania}
    \begin{itemize}
        \item Cel i zakres badania
        \item Przegląd literatury (min. 1 cytowanie powiązane tematycznie i krótki opis co było badane)
        \item Zmienne wybrane do analizy (opis i uzasadnienie zmiennych oraz podział na stymulanty/destymulanty) minimum sześć zmiennych
        \item Wstępna analiza danych
        \begin{itemize}
            \item statystyki opisowe (przynajmniej: średnia, mediana, minimum, maksimum, odchylenie standardowe, skośność)
                \begin{center}
                    \begin{tabular}{ |c|c|c|c|c|c|c|c|c|c| } 
                        \hline
                        & mpg & cylindry & objętość skokowa & KM & przyspieszenie & model & kraj \\
                        \hline
                        średnia & 20.762205  & 5.818898 & 216.090551 & 112.913386 & 3139.322835 & 15.313386 & 74.346457 & 1.464567 \\
                        \hline
                        min & 9 & 4 & 72 & 46 & 1613 & 9 & 70 & 1  \\
                        \hline
                        max & 43.1 & 8.0 & 455.0 & 230.0 & 5140.0 & 22.1 & 79.0 &  3.0  \\
                        \hline
                        odchylenie stan. & 6.6167528 & 1.7835995 & 109.9333567 & 41.6378993 & 900.7412457 & 2.8039757 & 2.9260135 &  0.7431729  \\
                        \hline
                        skośność & 0.71300125 & 0.17975327 & 0.33718586 & 0.79478074 & 0.24423388 & 0.06336346 & 0.04447328 &  1.21389161 \\
                    \hline
                    \end{tabular}
                \end{center}
            \item podstawowa wizualizacja np. boxplot, histogramy
            % braki danych, czy występują i jak je obsłużono
            \item Braki danych
            \item obserwacje odstające i w jaki sposób je obsłużono
        \end{itemize}
    \end{itemize}

\section{Opis metod}
    \begin{itemize}
        \item wzory wraz z opisami oznaczeń
        \item cytowanie pracy w której zaproponowano metodę/ewentualnie pracy, w której użyto metodę
    \end{itemize}

\section{Rezultaty (w postaci tabelarycznej i/lub graficznej oraz omówienie wyników)}
\section{Podsumowanie (ocena realizacji celu, odniesienie do pozycji z przeglądu literatury)}
\section{Bibliografia}
    \begin{itemize}
        \item Algorithm AS 136: A K-Means Clustering Algorithm - J. A. Hartigan and M. A. Wong
        \item www.statystyka.az.pl
        \item www.wikipedia.org
    \end{itemize}

\end{document}
